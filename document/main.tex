\documentclass{hw-template}
\usepackage[backend=bibtex]{biblatex}
\usepackage{csquotes}
\addbibresource{ref.bib}
\nocite{*}
\usepackage{hyperref}
\usepackage{subcaption}
\usepackage{lipsum}
\usepackage{tikz}
\usepackage{pgfplots}
\usepackage{minted}
\usepackage[shortlabels]{enumitem}
\usepackage{graphicx}
\usepackage{amsmath}
\usepackage{hyperref}


% Set lang to spanish
\usepackage[spanish]{babel}

\pgfplotsset{compat=1.18}

\newcommand*{\course}{Lenguajes Formales y Automatas (CM0081)}
\newcommand*{\assignment}{Laboratorio III}

\begin{document}

\maketitle
\begin{abstract}
    En este laboratorio se discute sobre la verificación de programas en
    representación intermedia de LLVM, el cual es un lenguaje de bajo nivel
    que se utiliza para representar programas de lenguajes de alto nivel,
    es un lenguaje que permite a los compiladores realizar optimizaciones
    de código. \\
    Se utiliza la herramienta formal \textit{VeLLVM} creada por
    \textit{DeepSpec} para verificar programas en representación intermedia
    de LLVM. \\
    Se discute la motivación del uso de las herramientas mencionadas, en 
    los lenguajes mencionados. \\ 
    Se discute la posibilidad de verificar programas que planteen problemas
    de teoría de números, ecuaciones diferenciales y grafos. \\
    Se discute la posibilidad de revisar y estudiar formalmente sobre la 
    implementación de verificación de programas hechos en ``Brainf**ck''
    usando Interaction Trees.
\end{abstract}

\tableofcontents

\clearpage

\section{Motivación}
Al momento de considerar la idea de verificar un programa, nos planteamos
la posibilidad de verificar código en ensamblador. Inicialmente, supusimos
que sería lógico comenzar con el lenguaje más bajo, y por ende, optamos
por ensamblador. Sin embargo, al reflexionar más a fondo, especialmente al
tener en cuenta que nuestro equipo es un `Macbook Pro M2` con una
arquitectura ARM, la cual difiere significativamente de la arquitectura x86
de Intel que conocemos, la idea de verificar ensamblador nos pareció 
'fuera de alcance'. Consideramos las múltiples arquitecturas de computadoras
existentes, cada una con sus propias instrucciones y diferencias, lo cual
complicaría el proceso. Esta línea de pensamiento también nos llevó a
contemplar la verificación de estructuras como 'FPGA (Field Programmable
Gate Array)' o 'ASIC (Application Specific Integrated Circuit)', ya que
habíamos trabajado con ellas en el pasado. Sin embargo, al igual que con
la idea de verificar ensamblador, encontramos desafíos significativos. \\

Al replantear nuestra perspectiva, recordamos un video de "MIT OpenCourseWare"
titulado "5. C to Assembly", donde se aborda la traducción del lenguaje C a
ensamblador. En este video, se menciona que el compilador "clang" utiliza
un lenguaje intermedio denominado "LLVM". Este último es un lenguaje de bajo
nivel diseñado para representar programas de lenguajes de alto nivel,
permitiendo a los compiladores realizar optimizaciones de código.

La idea cobró fuerza cuando nos percatamos de que múltiples lenguajes de
programación, como Rust \cite{rust}, C++ \cite{cpp}, Haskell \cite{haskell}
y muchos otros, emplean LLVM como representación intermedia para sus
compiladores. Por lo tanto, si es posible verificar un programa en LLVM,
también es posible hacerlo para Rust, C++, Haskell, entre otros. \\

\textit{\textbf{Nota:} Persistimos en la idea de realizar la verificación
de ensamblador. Descubrimos que diversas empresas,incluida Intel,
llevan a cabo la verificación formal de sus procesadores. Al revisar
detenidamente los documentos y presentaciones de estas empresas
\cite{intel-verification}, logramos obtener claridad sobre las posibilidades
y metodologías relacionadas con la verificación formal. Este proceso no
solo contribuyó a disipar incertidumbres, sino que también proporcionó un
enfoque más informado sobre qué se puede lograr y cómo llevar a cabo la
verificación formal de manera efectiva.} 

\section{Detalles Técnicos y versionamiento}

\subsection{Sistema Operativo y versiones}
Se utilizó el sistema operativo ``MacOS Sonoma'' versión 14.1.1, con la
versión del kernel ``Darwin23.1.0''. Se utilizó la versión de clang 
``clang-15.0.0'' y la versión de llvm ``llvm-16.0.0''. Se utilizó la
versión de OPAM ``2.1.5''. Se utilizó la versión de Coq ``8.15.2''. Se
utilizó la versión de Ocaml ``4.13.1''.  Se utilizó make ``3.81''.

\section{Instalación de herramientas}
Las instrucciones que se encuentran en el repositorio de VeLLVM 
\cite{vellvm} son muy claras, sin embargo hay que seguirlas en un orden
distinto al que se presentan, he aquí el orden para instalar VeLLVM:

\begin{enumerate}
    \item Instalar ``OPAM'' \cite{opam} (Ocaml Package Manager) con el gestor
    de paquetes en su sistema operativo, en mi caso ``Homebrew'' \cite{homebrew}
    con el comando ``brew install opam''.
    \item Inicializar ``OPAM'' con el comando ``opam init''.
    \item Agregar el repositorio de coq con el comando ``opam repo
    add coq-released https://coq.inria.fr/opam/released''.
    \item Crear un Switch para la instalación de VeLLVM con el comando 
    ``opam switch create vellvm ocaml-base-compiler.4.13.1''.
    \item Clonar el repositorio y los submodulos con el comando \\
    ``git clone {-}{-}recurse-submodules git@github.com:kurtcovayne/cm0081-2023-2-lab3.git''.
    \item Ingresar al directorio src con el comando ``cd cm0081-2023-2-lab3/vellvm/src''.
    \item Instalar las dependencias con el comando ``make opam''.
    \item Para compilar hubieron muchos problemas, pues tal parece que el make no está
    compilando las cosas en orden, lo que nos sirvió para arreglar esto fue
    ejecutar varias veces el comando ``make -j<n>'' donde ``n'' e ir aumentando
    el n en caso de que falle, en nuestro caso ``n'' llegó a ser 24.
\end{enumerate}

\section{Verificación de programas en representación intermedia de LLVM}

Para verificar un programa en representación intermedia de LLVM se debe
seguir los siguientes pasos:

\begin{enumerate}
    \item Crear un archivo con extensión ``.ll'' con el código en representación
    intermedia de LLVM.
    \item Compilar el archivo ``.ll'' con el comando ``clang -S -emit-llvm -o
    <nombre de archivo de salida> <nombre de archivo de entrada>''.
    \item Se tuvó problemas en ``MacOS Sonoma'' pues clang agrega modificadores
    a los parametros cuando ejecuta la instrucción call, también llvm agrega
    modificadores a los loops y VeLLVM no logra reconocerlos, por lo que se debe
    arreglar el archivo ``.ll'' para que no tenga estos modificadores.
    \item Agregar comandos de verificación al archivo ``.ll''
    \item Ejecutar el comando ``./vellvm --test-file <nombre de archivo de entrada>''.
\end{enumerate}

\section{Programas de interés}
Se verificaron los siguientes programas:
\begin{enumerate}
    \item Programa que realiza potenciación binaria de un número.
    \item Programa que calcula el gcd de dos números usando el algoritmo
    extendido de Euclides.
    \item Programa que calcula si existe una solución entera para la ecuación
    diofántica $ax + by = c$.
    \item Programa que calcula la multiplicación de dos números usando la transformada
    rápida de Fourier en un algoritmo conocido como ``Schönhage-Strassen algorithm''.
\end{enumerate}

Los programas en C se pueden encontrar en este \href{https://github.com/KurtCoVayne/cm0081-2023-2-lab3/tree/main/src}{repositorio} 

\section{Programas de futuro interes}
Será particularmente interesante abordar problemas de ecuaciones diferenciales,
donde se pueda verificar la solución para un espacio de entradas dado. Además,
se buscará extender la aplicación de la verificación formal a programas con
precondiciones sobre un grafo, permitiendo la formulación y verificación de
algoritmos que operan en este tipo de estructuras. \\

Algunos ejemplos concretos de programas futuros incluyen:

\begin{enumerate}
    \item Desarrollo de un programa que calcule la solución de una ecuación
    diferencial de primer orden, abarcando cualquier entrada dentro de un
    intervalo determinado.
    \item Implementación de un programa que encuentre el mínimo cubrimiento
    en un grafo, estableciendo precondiciones como la condición de ser un
    árbol o ser k-coloreable. Se buscará demostrar formalmente la terminación 
    y complejidad del algoritmo a nivel de implementación
\end{enumerate}

\section{Verificación de programas en Brainf**k}
En perspectiva, también se contempla la implementación de un programa de
verificación formal en el lenguaje Brainf**k, un lenguaje de programación
minimalista que se caracteriza por ser Turing completo. Este enfoque abrirá
la posibilidad de explorar la verificación formal en un contexto más
desafiante y peculiar, aprovechando las capacidades de este lenguaje para
expresar algoritmos de manera concisa. \\

La inclusión de programas en Brainf**k ampliará el alcance de la verificación
formal, ofreciendo una perspectiva única y desafiante en la validación 
rigurosa de algoritmos en lenguajes de programación menos convencionales.

\printbibliography
\end{document}